\chapter{Reinforcement Learning}

The neural network structures previously discussed for the Boston Housing data
are essentially curve fitting models, with continuous inputs and output.
The character recognition networks for the MNIST dataset can also be though of
as an abstract curve fitting model with multiple outputs, where the result is
a probability value.

In both cases, training is performed on a set of inputs with known target
outputs.

Not all problems can be constructed in this manner as some problems require
discrete actions to be taken.
Similarly, problems may contain reward systems.
Such systems may have delayed rewards, were one action provides an immediate
reward, but prevents reaching a state of higher reward.

Consider the ``Chain'' problem from \cite{Strens:2000:ABF}:
\rfig{\begin{tikzpicture}
    [   state/.style=
        {   draw
        ,   circle
        }
    ,   baseline=0pt
    ,   >={Stealth[scale=1.2]}
    ]
    \foreach\n in {0,...,4}
    {   \node[state] (N\n) at (2cm*\n, 0mm) {$\n$};
    }
    \foreach\np in {1,...,4}
    {   \pgfmathtruncatemacro{\nm}{\np-1}
        \draw[->]
        (N\np)
            to [out=225,in=315]
            node [near start, anchor = north west, inner sep = 2pt] {1}
        (N0);
        \draw[->]
        (N\nm)
            to [out=45,in=135]
            node [anchor = south, inner sep = 2pt] {0}
        (N\np);
    }
    \draw[->]
    (N0)
        .. controls ++(-1cm,-1cm) and ++(-1cm,1cm) ..
        node [anchor = east, inner sep = 2pt] {1}
    (N0);
    \draw[->]
    (N4)
        .. controls ++(1cm,1cm) and ++(1cm,-1cm) ..
        node [anchor = west, inner sep = 2pt] {10}
    (N4);
\end{tikzpicture}
}{%
    Five-state chain problem.
    Arrows from the top of the states denote action 0, and from the bottom
    denote action 1.
    The labels on each arrow denote the reward associated with that action.
    The system is initially at state 0.
}
The optimal, long term strategy for this system is to always perform action 0,
which will cause the system to reach state 4 and obtain a reward of 10 on each
step.
A greedy algorithm will discover the reward from always performing action 1,
which will cause the system to stay at state 0 and obtain a reward of 1 on each
step.
The greedy algorithm is only optimal for four steps or less.



\section{Q-Learning}

One way to find a solution to this problem without neural networks is
Q-Learning.
Proposed by \cite{Watkins:1992:Q}, Q-Learning is an algorithm for finding
optimal solutions to Markovian domain problems, such as the chain.

The algorithm starts with a table, referred to as the Q-table, which has a row
for each system state, and a column for each action, initialised as zero.
Once trained, each element of the Q-table will represent the maximum expected
reward for each action in each state.

Let $Q(s,a)$ denote the Q-table value for state $s$ and action $a$.

Initially, actions are taken at random, which allows the agent to explore the
network; but as training progresses, the agent gradually shifts towards making
decisions based on the Q-table values.

Each time an action $a$ is taken from state $s$, the reward $r$ and new state
$s'$ are observed, and the Q-table is updated using the equation
\begin{align*}
    Q(s,a) := Q(s,a) + \alpha (r + \gamma\max_{a'}Q(s',a') - Q(s,a)),
\end{align*}
where $\alpha$ is the learning rate, and $\gamma$ is the discount rate, which is
needed to prevent the values from growing exponentially.
Note that $\max_{a'}Q(s',a')$ is the maximum expected future reward of state
$s'$, according to the current Q-table values.

The algorithm was applied to the chain problem with learning rate $\alpha =
0.8$, and discount rate $\gamma = 0.7$, for 200 episodes, each consisting of 50
steps.
During training, the probability of the agent performing a random action was
given by $0.01^{t/200}$, where $t$ is the episode number.
\rfig{\begin{tikzpicture}
    \begin{axis}
        [   width = 0.8\textwidth
        ,   ymin = 0
        ,   ymax = 500
        ,   xtick = {0,40,80,120,160,200}
        ,   xlabel = {Episode}
        ,   ylabel = {Score}
        ]
        \addplot [color = Rblue]   table [col sep = comma]
        {../Code/RLChain/DataQLearning.csv};
    \end{axis}
\end{tikzpicture}
}{%
    Graph of score against episode number for the Q-Learning chain problem.
}
Once trained, the agent was able to obtain the highest possible score of 490.
The final values of the Q-table are provided below.
\rfig{%
    \begin{tabular}{c|c c}
        State & Action 0 & Action 1\\
        \hline
        0 &  8.00 & 6.60\\
        1 & 11.43 & 6.60\\
        2 & 16.33 & 6.60\\
        3 & 23.33 & 6.60\\
        4 & 33.33 & 6.60\\
    \end{tabular}}{%
    Q-table for the full trained Q-Learning chain problem.
}
Note that all of the values in the Action 0 column are larger than the
corresponding values in the Action 1 column, which denotes the larger expected
future reward.



\section{Deep Q-Learning}

The Q-Learning algorithm works well for problems with well defined states and
actions, but cannot accommodate problems where the state in not discrete.

To solve this problem,


\chapter{Introduction to Neural Networks}

\TODO{Chapter: Introduction}

\section{Biological Neurons}

Biological neurons are electrically excitable cells that are found in almost all
animals.
These neurons can transmit and receive electrical signals to one another via
synaptic connections, which maybe either excitatory or inhibitory.
Any given neuron will be either active or inactive depending on whether or not
its input exceeds a threshold.

\begin{center}
    \captionof{figure}{Diagram of a biological neuron.}
    \vspace{1ex}
    \begin{tikzpicture}%[x=1pt,y=1pt]
    \newlength{\rs}
    \setlength{\rs}{1.5pt}
    % Nucleus
    \fill (0\rs,0\rs) circle (2\rs);
    % Soma
    \draw[thick] (0\rs,0\rs) circle (8\rs);
    % Dendrite
    \draw[thick] ( 45:8\rs) -- ( 45:16\rs);
        \draw[thick] ( 45:16\rs) -- ++(  0:4\rs);
        \draw[thick] ( 45:16\rs) -- ++( 90:4\rs);
    \draw[thick] ( 90:8\rs) -- ( 90:24\rs);
        \draw[thick] ( 90:16\rs) -- ++( 45:4\rs);
        \draw[thick] ( 90:16\rs) -- ++(135:4\rs);
        \draw[thick] ( 90:24\rs) -- ++( 45:8\rs);
        \draw[thick] ( 90:24\rs) -- ++(135:8\rs);
    \draw[thick] (135:8\rs) -- (135:24\rs);
        \draw[thick] (135:16\rs) -- ++( 90:4\rs);
        \draw[thick] (135:16\rs) -- ++(180:4\rs);
        \draw[thick] (135:24\rs) -- ++( 90:8\rs);
        \draw[thick] (135:24\rs) -- ++(180:8\rs);
    \draw[thick] (180:8\rs) -- (180:24\rs);
        \draw[thick] (180:16\rs) -- ++(135:4\rs);
        \draw[thick] (180:16\rs) -- ++(225:4\rs);
        \draw[thick] (180:24\rs) -- ++(135:8\rs);
        \draw[thick] (180:24\rs) -- ++(225:8\rs);
    \draw[thick] (225:8\rs) -- (225:24\rs);
        \draw[thick] (225:16\rs) -- ++(180:4\rs);
        \draw[thick] (225:16\rs) -- ++(270:4\rs);
        \draw[thick] (225:24\rs) -- ++(180:8\rs);
        \draw[thick] (225:24\rs) -- ++(270:8\rs);
    \draw[thick] (270:8\rs) -- (270:24\rs);
        \draw[thick] (270:16\rs) -- ++(225:4\rs);
        \draw[thick] (270:16\rs) -- ++(315:4\rs);
        \draw[thick] (270:24\rs) -- ++(225:8\rs);
        \draw[thick] (270:24\rs) -- ++(315:8\rs);
    \draw[thick] (315:8\rs) -- (315:16\rs);
        \draw[thick] (315:16\rs) -- ++(270:4\rs);
        \draw[thick] (315:16\rs) -- ++(360:4\rs);
    % Axon
    \draw[thick] (8\rs,0\rs) -- (12\rs,0\rs);
    \draw[thick,rounded corners=3\rs] (12\rs,-3\rs) rectangle (24\rs,3\rs);
        \fill (18\rs,0\rs) circle (1.5\rs);
    \draw[thick] (24\rs,0\rs) -- (26\rs,0\rs);
    \draw[thick,rounded corners=3\rs] (26\rs,-3\rs) rectangle (38\rs,3\rs);
        \fill (32\rs,0\rs) circle (1.5\rs);
    \draw[thick] (38\rs,0\rs) -- (40\rs,0\rs);
    \draw[thick,rounded corners=3\rs] (40\rs,-3\rs) rectangle (52\rs,3\rs);
        \fill (46\rs,0\rs) circle (1.5\rs);
    \draw[thick] (52\rs,0\rs) -- (54\rs,0\rs);
    \draw[thick,rounded corners=3\rs] (54\rs,-3\rs) rectangle (66\rs,3\rs);
        \fill (60\rs,0\rs) circle (1.5\rs);
    \draw[thick] (66\rs,0\rs) -- (68\rs,0\rs);
    \draw[thick,rounded corners=3\rs] (68\rs,-3\rs) rectangle (80\rs,3\rs);
        \fill (74\rs,0\rs) circle (1.5\rs);
    % Axon terminal
    \draw[thick] (80\rs,0\rs) -- (82\rs,0\rs);
        \draw[thick] (82\rs,0\rs) -- (86\rs,4\rs);
            \draw[thick] (86\rs,4\rs) -- (86\rs,7\rs);
            \draw[thick] (86\rs,4\rs) -- (89\rs,4\rs);
        \draw[thick] (82\rs,0\rs) -- (86\rs,-4\rs);
            \draw[thick] (86\rs,-4\rs) -- (86\rs,-7\rs);
            \draw[thick] (86\rs,-4\rs) -- (89\rs,-4\rs);
    % Labels
    % Soma
    \draw[latex'-] (65:9\rs) -- (65:36\rs);
    \node[anchor=south west,inner sep=1\rs] at (65:36\rs) {Soma};
    % Nucleus
    \draw[latex'-] (-65:3\rs) -- (-65:36\rs);
    \node[anchor=north west,inner sep=1\rs] at (-65:36\rs) {Nucleus};
    % Dendrite
    %\draw[latex'-] (225:25\rs) -- (225:36\rs);
    \draw[latex'-] (-17.7\rs,-25.7\rs) -- (-30\rs,-38\rs);
    \node[anchor=north east,inner sep=1\rs] at (-30\rs,-38\rs) {Dendrite};
    % Axon
    \draw[decorate,decoration={brace,raise=5\rs,amplitude=4\rs}] (12\rs,0\rs) -- (80\rs,0\rs);
    \node[anchor=south,inner sep=1\rs] at (46\rs,9\rs) {Axon};
    % Axon Terminal
    \draw[latex'-] (87\rs,8\rs) -- (95\rs,16\rs);
    \node[anchor=south west,inner sep=1\rs] at (95\rs,16\rs) {Axon Terminal};
\end{tikzpicture}

\end{center}

Signals are received by the neuron via connections to dendrites and soma.
If the threshold is met, electrical signals are sent along the axon to the
terminal, where it is connect to more neurons or to a controllable cell such as
a neuromuscular junction.

\TODO{Section: Biological Neurons}

\section{Artificial Intelligence}

The idea of artificial beings capable of human intelligence can be traced back
to mythical stories from ancient Greece.
One such story was that of a mythical automaton called Talos, who circled an
island's shores to protect it from pirates and other invaders.

By the $19^\text{th}$ century, other notions of artificial intelligence were
explored by fiction in stories, such as Mary Shelley's ``Frankenstein'', and
Karel \v{C}apek's ``R.U.R.''.
Some of the fictional writings of the $20^\text{th}$ century further continued
to explore the concept in novels such as Isaac Asimov's ``I, Robot''.

\TODO{Section: Artificial Intelligence}

\subsection{Perceptrons}

The idea of the perceptron was originally conceived by
\cite{Rosenblatt:1958:Perceptron}, to represent a simplified model of
intelligent systems free from particularities of biological organisms, whilst
maintaining some of their fundamental properties.

The perceptron was built as a dedicated machine that consisted of a number of
photovoltaic, analogous to a retina, that feed into an ``association area''.
This association area contains a number of cells that each calculate a weighted
sum of the receptor values and output a signal if it exceeds a threshold.
These value weights were implemented using variable resistance wires that the
perceptron could adjust automatically.
The outputs from the association area are then connected to response cells,
which operate in a similar fashion to the association cells.
The activation of these response cells are the outputs of the perceptron, and
indicated the classification of the input.

This machine was initially trained to reliably identify three different shapes:
a square, a circle, and a triangle; and did so with a better than chance
probability.
When attempting to use the perceptron for more complicated tasks, such as
character recognition, it failed to produce better than chance results.

\TODO{Subsection: Perceptrons}

\subsection{Backpropagation}

\TODO{Subsection: Backpropagation}

\section{Types of Neurons}

\TODO{Section: Types of Neurons}

\subsection{CNN}
\subsection{RNN}
\subsection{LSTM}
